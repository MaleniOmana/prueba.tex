\documentclass{article}

\usepackage[utf8]{inputenc}
\usepackage[spanish]{babel}
\usepackage{amsmath}

\begin{document}
1. Escribela solucion de la ecuacion en diferencias $x_{n+2}-4x_n=0$
con condiciones iniciales $x_{0}=1  x_{0}=-1$
Solucion:
La ecuacion resolvente es $r^{2}-4=0 r=\pm\2$ 
$r_1=2$ $r_2=-2$
$x_n=\alpha_1{2^n}+\alpha_2{-2^n}$
Como $1=x_0=\alpha_1+\alpha_2$  y $-1=x_1=2\alpha_1-2\alpha_2$
Hallando las soluciones de las ultimas dos ecuaciones se sigue que:
$\alpha_1=\frac{1}{4}$ $\alpha_2=\frac{3}{4}$
por tanto $x_n=\frac{1}{4}2^n-\frac{3}{4}2^n$
\end{document}
