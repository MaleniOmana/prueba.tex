\documentclass{article}

\usepackage[utf8]{inputenc}
\usepackage[spanish]{babel}
\usepackage{amsmath}

\begin{document}
Ejercicios:
\begin{enumerate}
\item 

Escribe la solucion de la ecuacion en diferencias $x_{n+2}-4x_n=0$
con condiciones iniciales $x_{0}=1  x_{0}=-1$
Solucion:
La ecuacion resolvente es:

\begin{align*}
  r^{2}-4&=0\\
  r&=\pm 2\\ 
  r_1&=2\\
  r_2&=-2\\
  x_n&=\alpha_1{2^{n}}+\alpha_2{-2^{n}}
\end{align*}
\item
  
Como $1=x_0=\alpha_1+\alpha_2$  y
$-1=x_1=2\alpha_1-2\alpha_2$
Hallando las soluciones de las ultimas dos ecuaciones se sigue que:
$\alpha_1=\frac{1}{4}$ $\alpha_2=\frac{3}{4}$
por tanto $x_n=\frac{1}{4} 2^{n}-\frac{3}{4} -2^{n}$
2. Si $d_n=ndn_1+(-1)^n$ para todo $n>=1$ y se tiene la condicion inicial $d_0=0$ calcula $d_4$
Solucion:
$d_0=0$
\begin{align*}
  d_1&=1(d_0)+(-1)\\
  d_2&=2(d_1)+(-1)^2\\
  d_3&=3(d_2)+(-1)^3\\
  d_4&=4(d_3)+(-1)^4\\
  d_4&=-15
\end{align*}
\end{enumerate}
\end{document}
